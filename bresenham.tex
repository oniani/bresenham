%%%%%%%%%%%%%%%%%%%%%%%%%%%%%%%%%%%%%%%%%%%%%%%%%%%%%%%%%%%%%%%%%%%%%%%%%%%%%%%
%
% Author: David Oniani
%
%  _         _____   __  __
% | |    __ |_   _|__\ \/ /
% | |   / _` || |/ _ \\  /
% | |__| (_| || |  __//  \
% |_____\__,_||_|\___/_/\_\
%
%%%%%%%%%%%%%%%%%%%%%%%%%%%%%%%%%%%%%%%%%%%%%%%%%%%%%%%%%%%%%%%%%%%%%%%%%%%%%%%

%%%%%%%%%%%%%%%%%%%%%%%%%%%%%%%%%%%%%%%%%%%%%%%%%%%%%%%%%%%%%%%%%%%%%%%%%%%%%%%
% Document Definition
%%%%%%%%%%%%%%%%%%%%%%%%%%%%%%%%%%%%%%%%%%%%%%%%%%%%%%%%%%%%%%%%%%%%%%%%%%%%%%%

\documentclass{article}

%%%%%%%%%%%%%%%%%%%%%%%%%%%%%%%%%%%%%%%%%%%%%%%%%%%%%%%%%%%%%%%%%%%%%%%%%%%%%%%
% Packages and Related Settings
%%%%%%%%%%%%%%%%%%%%%%%%%%%%%%%%%%%%%%%%%%%%%%%%%%%%%%%%%%%%%%%%%%%%%%%%%%%%%%%

% Global, document-wide settings
\usepackage[margin=1in]{geometry}
\usepackage[utf8]{inputenc}
\usepackage[english]{babel}

% Other packages
\usepackage{caption}
\usepackage{hyperref}
\usepackage{mathtools}
\usepackage{minted}
\usepackage{xcolor}
\usepackage{sectsty}

%%%%%%%%%%%%%%%%%%%%%%%%%%%%%%%%%%%%%%%%%%%%%%%%%%%%%%%%%%%%%%%%%%%%%%%%%%%%%%%
% Setup
%%%%%%%%%%%%%%%%%%%%%%%%%%%%%%%%%%%%%%%%%%%%%%%%%%%%%%%%%%%%%%%%%%%%%%%%%%%%%%%

% Remove indentations from paragraphs
\setlength{\parindent}{0pt}

% Orange color
\definecolor{orange}{HTML}{f0360e}

% Black-blue color
\colorlet{bb}{black!50!blue}

% Change section color
\sectionfont{\color{orange}}

% PDF information and nice-looking urls
\hypersetup{%
    pdfauthor={David Oniani},
    pdftitle={Bresenham's Circle Drawing Algorithm},
    pdfsubject={circle, geometry, bresenham},
    pdfkeywords={circle, geometry, bresenham},
    pdflang={English},
    colorlinks=true,
    linkcolor={bb},
    citecolor={bb},
    urlcolor={bb}
}

%%%%%%%%%%%%%%%%%%%%%%%%%%%%%%%%%%%%%%%%%%%%%%%%%%%%%%%%%%%%%%%%%%%%%%%%%%%%%%%
% Author(s), Title, and Date
%%%%%%%%%%%%%%%%%%%%%%%%%%%%%%%%%%%%%%%%%%%%%%%%%%%%%%%%%%%%%%%%%%%%%%%%%%%%%%%

% Author(s)
\author{David Oniani\\
        \href{mailto:onianidavid@luther.edu}{onianidavid@luther.edu}}

% Title
\title{\textcolor{orange}{\textit{Bresenham's Circle Drawing Algorithm}}}

% Date
\date{\today}

%%%%%%%%%%%%%%%%%%%%%%%%%%%%%%%%%%%%%%%%%%%%%%%%%%%%%%%%%%%%%%%%%%%%%%%%%%%%%%%
% The Beginning of the Document
%%%%%%%%%%%%%%%%%%%%%%%%%%%%%%%%%%%%%%%%%%%%%%%%%%%%%%%%%%%%%%%%%%%%%%%%%%%%%%%

\begin{document}
\maketitle

%%%%%%%%%%%%%%%%%%%%%%%%%%%%%%%%%%%%%%%%%%%%%%%%%%%%%%%%%%%%%%%%%%%%%%%%%%%%%%%
% Derivation
%%%%%%%%%%%%%%%%%%%%%%%%%%%%%%%%%%%%%%%%%%%%%%%%%%%%%%%%%%%%%%%%%%%%%%%%%%%%%%%

\section{Derivation}

Let the radius of the circle be given and equal \(r\). Any point \((x, y)\) on
the circle satisfies the following:
\begin{equation}
    x^2 + y^2 = r^2
\end{equation}

We consider the upper 45 degrees in the first quadrant of the circle. We can
place the next point either to the east \((x + 1, y)\) or southeast \((x + 1, y
- 1)\). We define errors for both of these points:
\begin{align}
    &e(x + 1, y) = {(x + 1)}^2 + y^2 - r^2\\
    &e(x + 1, y - 1) = {(x + 1)}^2 + {(y - 1)}^2 - r^2
\end{align}

We now define the decision parameter \(d\):
\begin{equation}
    d = e(x + 1, y) + e(x + 1, y - 1) = 2{(x + 1)}^2 + y^2 + {(y - 1)}^2 - 2r^2
\end{equation}

Since the \(x\) coordinate always increases, the next point is \((x + 1, y_n)\)
and we have:
\begin{equation}
    d_n = 2{(x + 2)}^2 + {y_n}^2 + {(y_n - 1)}^2 - 2r^2
\end{equation}

Now, let us calculate the difference:
\begin{align}
    &d_n - d = 2{(x + 2)}^2 + {y_n}^2 + {({y_n} - 1)}^2 - 2r^2 -
        (2{(x + 1)}^2 + {y}^2 + {(y - 1)}^2 - 2r^2)\\
    &d_n - d = 2(2x + 3) + \big({y_n}^2 - y^2\big) +
        \big({(y_n - 1)}^2 - {(y - 1)}^2\big)
\end{align}

Finally, we get:
\begin{align}
    &d_n - d = 2(2x + 3) &&\implies d_n = d + 4x + 6
        &&\text{ if } d \leq 0\ (y_n = y)\\
    &d_n - d = 2(2x + 3) - 4y + 4 &&\implies d_n = d + 4(x - y) + 10
        &&\text{ if } d > 0\ (y_n = y - 1)\\
    &d_0 = 2{(0 + 1)}^2 + r^2 + {(r - 1)}^2 - 2r^2 &&\implies d_0 = 3 - 2r
        &&\text{ at initial point } (0, r)
\end{align}

%%%%%%%%%%%%%%%%%%%%%%%%%%%%%%%%%%%%%%%%%%%%%%%%%%%%%%%%%%%%%%%%%%%%%%%%%%%%%%%
% Implementation
%%%%%%%%%%%%%%%%%%%%%%%%%%%%%%%%%%%%%%%%%%%%%%%%%%%%%%%%%%%%%%%%%%%%%%%%%%%%%%%

\clearpage
\section{Implementation}

\begin{itemize}
    \item Time Complexity: \(O(r)\) where \(r\) is the radius of the circle
    \item Space Complexity: \(O(1)\)
\end{itemize}

\begin{figure}[H]
    \inputminted[fontsize=\small,frame=single,framerule=1pt]{python}{bresenham.py}
    \caption*{A Python Implementation of Bresenham's Circle Drawing Algorithm}
\end{figure}

%%%%%%%%%%%%%%%%%%%%%%%%%%%%%%%%%%%%%%%%%%%%%%%%%%%%%%%%%%%%%%%%%%%%%%%%%%%%%%%
% The End of the Document
%%%%%%%%%%%%%%%%%%%%%%%%%%%%%%%%%%%%%%%%%%%%%%%%%%%%%%%%%%%%%%%%%%%%%%%%%%%%%%%

\end{document}
